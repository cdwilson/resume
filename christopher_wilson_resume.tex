%% christopher_wilson_resume.tex
%% Copyright 2021 Christopher Wilson (christopher.david.wilson@gmail.com)
%
% This work may be distributed and/or modified under the
% conditions of the LaTeX Project Public License, either version 1.3
% of this license or (at your option) any later version.
% The latest version of this license is in
%   http://www.latex-project.org/lppl.txt
% and version 1.3 or later is part of all distributions of LaTeX
% version 2005/12/01 or later.
%
% This work has the LPPL maintenance status `maintained'.
%
% The Current Maintainer of this work is Christopher Wilson
%
% This work consists of the files christopher_wilson_resume.tex

\documentclass[11pt,a4paper]{moderncv}

\moderncvstyle{classic}
\moderncvcolor{blue}

\usepackage[utf8]{inputenc}

\usepackage[top=1.1cm, bottom=1.1cm, left=2cm, right=2cm]{geometry}
\setlength{\hintscolumnwidth}{2.5cm}

\name{Chris}{Wilson}
% \title{}
\address{Castro Valley, CA}
\phone[mobile]{+1~(858)~722~2298}
\email{christopher.david.wilson@gmail.com}
\homepage{cdwilson.dev}
\social[linkedin]{cdwilson}
\social[github]{cdwilson}

% https://tex.stackexchange.com/questions/5036/how-to-prevent-latex-from-hyphenating-the-entire-document/5039
\tolerance=1
\emergencystretch=\maxdimen
\hyphenpenalty=10000
\hbadness=10000

\makeatletter
\AtBeginDocument{
  \include{revision}
  \hypersetup{
    pdftitle = {\@firstname{}~\@lastname{} -- \@title{} -- Rev \Revision},
    pdfsubject = {R\'{e}sum\'{e} of \@firstname{}~\@lastname{}},
  }
}
\makeatother

\begin{document}

\makecvtitle

\section{Vocational Experience}
\cventry{2022}{Technical Advisor}{\httpslink[Tempo Automation]{www.tempoautomation.com}}{San Francisco, CA}{}{}
\cventry{2018 -- 2021}{Senior Technical Product Manager}{}{}{}{}
\cventry{2018}{Technical Product Manager}{}{}{}{
  \textit{Software-Accelerated PCBA Manufacturing}
  \begin{itemize}
    \item Conducted user, market, and competitive research to identify product opportunities in the customer experience and inform the product roadmap communicated to c-level leadership.
    \item Responsible for the design, planning, execution, and launch of real-time order status tracking and PCB design visualization features in the customer portal, contributing to a 19\% increase in NPS from 2018 to 2019.
    \item Owned an initiative to reduce the time-to-RFQ by redesigning the bill of materials (BOM) editor, resulting in a 23\% reduction in the median BOM issue resolution time.
    \item Established technical credibility with key customers by participating in IPC-2581 technical committee meetings, ultimately leading to investment from Lockheed Martin (Series C).
  \end{itemize}
}
\cventry{2010 -- 2018}{Electronics Design Engineer}{\httpslink[Cisco Systems]{www.cisco.com}}{San Jose, CA}{}
{
  \textit{Industrial Internet of Things (IIoT) solutions for Smart Grid}
  \begin{itemize}
    \item Co-designed Cisco's first industrial IOx ``fog'' compute module, enabling customers to run custom IoT applications on Cisco 1000 Series Connected Grid Routers.
    \item Lead electronics design engineer for IEEE 802.15.4g hardware reference designs used by Cisco DevNet partners to develop 3rd-party Connected Grid Endpoint (CGE) devices.
    \item Developed the world's largest closed-circuit mesh network testbed consisting of over 5000 IoT hardware endpoint devices, unlocking CI/CD workflows and remote development/debug/testing for internal firmware development teams.
  \end{itemize}
}
\cventry{2007 -- 2010}{Electronics Design Engineer}{\httpslink[Arch Rock (acquired by Cisco Systems)]{www.cisco.com/c/en/us/about/corporate-strategy-office/acquisitions/arch-rock.html}}{San Francisco, CA}{}
{
  \textit{Pioneer in IP-based wireless sensor network technology}
  \begin{itemize}
    \item Responsible for transition to agile in-house hardware design and manufacturing.  Adopted industry standard EDA, DFM, PLM tools and methodology to scale hardware development from prototype to production.
    \item Designed and launched 802.15.4 2.4GHz PhyNet™ wireless sensors and router network interface cards for enterprise-scale wireless sensor networks.
  \end{itemize}
}
\section{Personal Projects}
\cventry{2010 -- 2013}{Owner}{\httpslink[Flying Camp Design]{flyingcamp.design}}{Castro Valley, CA}{}
{
  \textit{Open-source hardware design}
  \begin{itemize}
    \item Designed open-source hardware boot-strap loader (BSL) programmer for TI MSP430 MCUs.
    \item Developed open-source cross-platform BSL GUI utility in Python.
  \end{itemize}
}

\section{Skills \& Interests}
\cvitem{Product}{Jira, Confluence, Git/Github, Python}
\cvitem{Electronics}{KiCad, Cadence Concept \& Allegro, OrCAD Capture, Autodesk EAGLE, Arena PLM, Oracle Agile PLM, lab safety, PCBA bring-up \& rework}
\cvitem{Interests}{Embedded systems, traveling, mountain biking, surfing}

\section{Education}
\cventry{2003 -- 2007}{B.S. Electrical Engineering and Computer Science}{\httpslink[University of California Berkeley]{www.berkeley.edu}}{Berkeley, CA}{}{Awards: Edward Frank Kraft Scholarship}

\end{document}
