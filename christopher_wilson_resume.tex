%% christopher_wilson_resume.tex
%% Copyright 2014 Christopher Wilson (cwilson@cdwilson.us)
%
% This work may be distributed and/or modified under the
% conditions of the LaTeX Project Public License, either version 1.3
% of this license or (at your option) any later version.
% The latest version of this license is in
%   http://www.latex-project.org/lppl.txt
% and version 1.3 or later is part of all distributions of LaTeX
% version 2005/12/01 or later.
%
% This work has the LPPL maintenance status `maintained'.
%
% The Current Maintainer of this work is Christopher Wilson
%
% This work consists of the files christopher_wilson_resume.tex

\documentclass[11pt,a4paper]{moderncv}

\moderncvstyle{classic}
\moderncvcolor{blue}

\usepackage[utf8]{inputenc}

\usepackage[top=1.1cm, bottom=1.1cm, left=2cm, right=2cm]{geometry}
\setlength{\hintscolumnwidth}{2.5cm}

\name{Christopher}{Wilson}
\title{Technical Product Manager}
\address{5645 Cold Water Dr.}{Castro Valley, CA 94552}
\phone[mobile]{+1~(858)~722~2298}
\email{cwilson@cdwilson.us}
\homepage{www.cdwilson.us}
\social[linkedin]{cdwilson}
\social[github]{cdwilson}

\makeatletter
\AtBeginDocument{
  \include{revision}
  \hypersetup{
    pdftitle = {\@firstname{}~\@lastname{} -- \@title{} -- Rev \Revision},
    pdfsubject = {R\'{e}sum\'{e} of \@firstname{}~\@lastname{}},
  }
}
\makeatother

\begin{document}

\makecvtitle

\section{Experience}
\subsection{Vocational}
\cventry{2018 -- Present}{Technical Product Manager}{Tempo Automation}{San Francisco, CA}{}{}
\cventry{2010 -- 2018}{Hardware Design Engineer}{Cisco Systems}{San Jose, CA}{}
{
  \textit{Internet of Things (IoT) BU}
  \begin{itemize}
  \item Lead design engineer for Connected Grid Endpoint (CGE) SDK hardware reference design.
    \begin{itemize}
      \item Designed ARM Cortex-M \& Tensilica based modular reference designs for IEEE 802.15.4g 800/900 MHz RF and 1901.2 PLC smart grid endpoint devices.
      \item Managed external ECAD and MCAD contractors during PCB layout and enclosure design.
      \item Responsible for NPI engineering using Cisco manufacturing tools.
      \item Worked closely with Cisco Developer Network (CDN) partners to review partner hardware designs for "Cisco Compatible" certification.
    \end{itemize}
  \item Co-designed Cisco's first industrial "Fog" compute server module, allowing customers to utilize Cisco IOx hypervisor architecture to run Internet of Things (IoT) applications inside CGR 1000 series routers.
    \begin{itemize}
      \item Designed interface logic boards for AMD G-Series COM Express module to meet stringent worldwide and smart grid performance requirements for harsh operating environments.
      \item Validated HW design specifications over industrial operating temperature range, and performed physical layer qualification testing for SATA, Gigabit Ethernet, and USB interfaces.
      \item Worked with internal ECAD and MCAD teams to complete PCB layout and chassis design.
      \item Guided operations team to transition the product from NPI to production.
    \end{itemize}
  \item Developed the worlds largest closed-circuit RF and PLC mesh network testbed consisting of over 5000 endpoint devices.
    \begin{itemize}
    \item Designed sophisticated custom rack-mount chassis and internal backplane PCB with embedded Linux ARM controller, providing networked back-channel and instruction-level JTAG/SWD debug to every endpoint in the testbed.
    \item Modified Linux kernel source code and BSP to support custom backplane PCB hardware.
    \item Developed user space applications in C for CLI device control.
    \item Developed Python CGI web application for device management from a web browser.
    \end{itemize}
  \item Managed DevOps for the Connected Grid Endpoint firmware team.
    \begin{itemize}
    \item Set up and maintained Git (SCM), Jenkins (CI and release build), and Gerrit (code review) servers.
    \item Developed build scripts in Windows batch and Bash for a hybrid Cygwin/IAR firmware build server.
    \end{itemize}
  \end{itemize}
}
\cventry{2007 -- 2010}{Hardware Engineer}{Arch Rock (acquired by Cisco Systems)}{San Francisco, CA}{}
{
  \begin{itemize}
    \item Responsible for transition to agile in-house hardware design and manufacturing.  Adopted industry standard EDA, DFM, and PLM tools and methodology to scale hardware development for production.
    \item Designed 802.15.4 2.4GHz “PhyNet” wireless sensor motes and network interface cards.
      \begin{itemize}
        \item Schematic capture and PCB layout using OrCAD Capture and Cadence Allegro.
        \item Board level bring-up and verification using lab test equipment.
        \item Hand assembled and reworked prototype PCBs.
      \end{itemize}
    \item Developed embedded TinyOS firmware applications in nesC (network embedded systems C) for hardware bring-up and manufacturing test.  Debugged and optimized production runtime firmware with special emphasis on low power operation.
    \item Designed and built a fully isolated wireless mesh testbed with reconfigurable RF topologies.
    \begin{itemize}
      \item Developed integration test scripts in Ruby to allow automated deployment of embedded firmware for devices in the testbed.
      \item Tightly integrated the testbed with Buildbot based continuous integration to facilitate automated regression testing for the full mesh network stack.
    \end{itemize}
  \end{itemize}
}
\cventry{2006}{Undergraduate Researcher}{Berkeley Wireless Research Center}{Berkeley, CA}{}
{
  \begin{itemize}
    \item Implemented distributed adaptive duty cycling algorithm in nesC for Telos wireless sensor motes running TinyOS 1.x operating system.
  \end{itemize}
}
\cventry{2004}{Interim Engineering Intern}{Qualcomm}{San Diego, CA}{}
{
  \textit{CDMA Technologies form factor accurate (FFA) baseband team}
  \begin{itemize}
    \item Developed framework for an intranet website used to track internal development of FFA hardware.
  \end{itemize}
}
\subsection{Hobby}
\cventry{2010 -- Present}{Proprietor}{Flying Camp Design}{Castro Valley, CA}{}
{
  \textit{Indie hardware and software design}
  \begin{itemize}
    \item Designed open source hardware boot-strap loader (BSL) programmer for TI MSP430 MCUs.
    \item Manufactured and sold over 100 programmers internationally.
    \item Developed open source cross-platform BSL GUI utility in Python.
  \end{itemize}
}
\cventry{2010 -- 2015}{Partner}{Moteware}{Berkeley, CA}{}
{
  \textit{Open source hardware disseminator for the academic research community}
  \begin{itemize}
    \item Founded with a group of former graduate students at UC Berkeley.
    \item Helped manage sales, support, IT, and manufacturing.
  \end{itemize}
}

\section{Education}
\cventry{2003 -- 2007}{B.S. Electrical Engineering and Computer Science}{University of California, Berkeley}{Berkeley, CA}{}{Awards: Edward Frank Kraft Scholarship\newline{}Activities and Societies: IEEE, Alpha Gamma Omega, FoCUS}

\section{Skills \& Expertise}
\cvitem{Lab}{PCA bring-up, electronic test equipment, hand soldering/rework, lab safety}
\cvitem{EDA}{Cadence Concept and Allegro, OrCAD Capture, Cadsoft EAGLE}
\cvitem{PLM}{Cisco Agile, Arena Solutions}
\cvitem{Basic}{Git, Python, make, C, nesC, Java, Bash, Ruby, Tcl/Tk, \LaTeX, Bazaar, SVN, CVS}
\cvitem{Environment}{Mac OS X, Linux, Windows, Cygwin}

\section{Interests}
\cvitem{Travel}{Traveled extensively in Europe and parts of Africa.}
\cvitem{Social Justice}{Projects in Kenya, Haiti, and Mexico.}
\cvitem{Sports}{Surfing, skateboarding, snowboarding, biking, lacrosse}

% \makecvfooter{test}
% \newlength{\footerwidth}
% \newlength{\footerboxwidth}%
% \setlength{\footerwidth}{0.8\textwidth}%
% \fancypagestyle{plain}{
%   \fancyfoot[c]{
%     \parbox[b]{0.8\textwidth}{
%       \centering
%       \color{color2}
%       \vspace{\baselineskip}
%       \vspace{-\baselineskip}
%       \strut{\bfseries\upshape\@firstname~\@lastname}
%       \addtofooter[]{test}
%     }
%   }
% }
% \ifthenelse{\lengthtest{\footerboxwidth=0pt}}{}{\flushfooter}
% \pagestyle{plain}
% \fancyfoot[c]{%
%   \parbox[b]{0.8\textwidth}{%
%     \centering%
%       \vspace{\baselineskip}% forces a white line to ensure space between main text and footer (as footer height can't be known in advance)
%       \vspace{-\baselineskip}% to cancel out the extra vertical space taken by the name (below) and ensure perfect alignment of letter and cv footers
%       \strut{\bfseries\upshape firstname~lastname}\\% the \strut is required to ensure the line is exactly \baselineskip tall
%       street number~city\\
%       phone~email~etc.
%   }%end parbox
% }%end fancyfoot

\end{document}
